\documentclass[10pt]{beamer}
\usepackage[utf8]{inputenc}
\usepackage[T1]{fontenc}
\usetheme{default}
\usetheme{bars}
\usetheme{Madrid}
\usepackage{multicol}
\usepackage[absolute,overlay]{textpos}
\usepackage{xcolor,colortbl,color}
\usepackage{graphicx,graphics,epsf,rotate,wrapfig}
\usepackage{pgf,pgfarrows,pgfnodes,pgfautomata,pgfheaps, pgfpages}
\usepackage{latexsym,amssymb,amsmath,hyperref, longtable,mathrsfs,euscript}
\usepackage{color}
\usepackage{wrapfig}

\usepackage{xcolor}



\usepackage{listings}
\usepackage{xcolor}

\definecolor{codegreen}{rgb}{0,0.6,0}
\definecolor{codegray}{rgb}{0.5,0.5,0.5}
\definecolor{codepurple}{rgb}{0.58,0,0.82}
\definecolor{backcolour}{rgb}{0.95,0.95,0.92}

\lstdefinestyle{mystyle}{
	language=Python,
	backgroundcolor=\color{backcolour},   
	commentstyle=\color{codegreen},
	keywordstyle=\color{magenta},
	numberstyle=\tiny\color{codegray},
	stringstyle=\color{codepurple},
	basicstyle=\ttfamily\footnotesize,
	breakatwhitespace=false,         
	breaklines=true,                 
	captionpos=b,                    
	keepspaces=true,                 
	numbers=left,                    
	numbersep=5pt,                  
	showspaces=false,                
	showstringspaces=false,
	showtabs=false,                  
	tabsize=2
}


\lstset{style=mystyle}




\setbeamertemplate{theorems}[numbered] 
\newtheorem{remark}{Remark} %for remark 

\renewcommand\thetheorem{\arabic{lecture}.\arabic{theorem}}
\renewcommand\theremark{\arabic{lecture}.\arabic{remark}}
\makeatletter
\@addtoreset{theorem}{lecture}
\@addtoreset{remark}{lecture}

%\AtBeginEnvironment{align}{\setcounter{equation}{0}}

%\DeclareMathOperator{\rank}{rank}
\DeclareMathOperator{\esssup}{ess\,sup}


%\theoremstyle{plain}
%\newtheorem{theorem}{Theorem}[section]
%\newtheorem{lemma}[theorem]{Lemma}
%\newtheorem{corollary}[theorem]{Corollary}
\newtheorem{proposition}[theorem]{Proposition}
\newtheorem{conjecture}[theorem]{Conjecture}
\newtheorem{criterion}[theorem]{Criterion}
\newtheorem{algorithm}[theorem]{Algorithm}

\theoremstyle{definition}
%\newtheorem{definition}[theorem]{Definition}
\newtheorem{condition}[theorem]{Condition}
%\newtheorem{problem}[theorem]{Problem}
%\newtheorem{example}[theorem]{Example}
\newtheorem{assumption}[theorem]{Assumption}
\theoremstyle{remark}
%\newtheorem{remark}[theorem]{Remark}
%\newtheorem{note}[theorem]{Note}
\newtheorem{notation}[theorem]{Notation}
\newtheorem{claim}[theorem]{Claim}
\newtheorem{summary}[theorem]{Summary}
\newtheorem{acknowledgment}[theorem]{Acknowledgment}
\newtheorem{case}[theorem]{Case}
\newtheorem{conclusion}[theorem]{Conclusion}
%%%%%Some shortcuts for sets%%%%%
\newcommand{\R}{\mathbb{R}}
\newcommand{\N}{\mathbb{N}}
\newcommand{\Z}{\mathbb{Z}}
\newcommand{\Q}{\mathbb{Q}}
\newcommand{\p}{\partial}
\newcommand{\A}{\mathcal{A}}
\newcommand{\F}{\mathcal{F}}
\newcommand{\B}{\mathcal{B}}
\newcommand{\M}{\mathcal{M}}
\newcommand{\C}{\mathcal{C}}


\newcommand{\bld}{\boldsymbol}
\newcommand{\bu}{\boldsymbol{u}}
\newcommand{\bx}{\boldsymbol{x}}
\newcommand{\bv}{\boldsymbol{v}}
\newcommand{\bw}{\boldsymbol{w}}
\newcommand{\bphi}{\boldsymbol{\varphi}}
\newcommand{\bF}{\boldsymbol{F}}
\newcommand{\bg}{\boldsymbol{g}}
\newcommand{\bPhi}{\boldsymbol{\phi}}
\newcommand{\bpsi}{\boldsymbol{\psi}}
%\newcommand{\nabla}{\boldsymbol{\nabla}}
\newcommand{\ltwo}[1]{{L^2\left(#1\right)}}
\newcommand{\lfour}[1]{{L^4\left(#1\right)}}
\newcommand{\linf}[1]{{L^\infty\left(#1\right)}}
\newcommand{\hdot}[1]{{\dot H^1\left(#1\right)}}
\newcommand{\hdotzig}[1]{{\dot H^1_{0,\sigma}\left(#1\right)}}
\newcommand{\hdotz}[1]{{\dot H^1_0\left(#1\right)}}
\newcommand{\upper}{{\R^2_+}}
\newcommand{\balln}{{B^+_n}}
\newcommand{\ballm}{{B^+_m}}
\newcommand{\com}[1]{{C^\infty_c\left(#1\right)}}
\newcommand{\comsig}[1]{{C^\infty_{c,\sigma}\left(#1\right)}}

\usepackage{subfigure}
%\usepackage{enumitem}
\newcounter{nameOfYourChoice}
\def\Xint#1{\mathchoice
	{\XXint\displaystyle\textstyle{#1}}%
	{\XXint\textstyle\scriptstyle{#1}}%
	{\XXint\scriptstyle\scriptscriptstyle{#1}}%
	{\XXint\scriptscriptstyle\scriptscriptstyle{#1}}%
	\!\int}
\def\XXint#1#2#3{{\setbox0=\hbox{$#1{#2#3}{\int}$ }
		\vcenter{\hbox{$#2#3$ }}\kern-.6\wd0}}
\def\ddashint{\Xint=}
\def\dashint{\Xint-}

\newcommand{\dist}{{\rm dist }}
\newcommand{\usub}{\underline{u}}
\newcommand{\usup}{\overline{u}}
\newcommand{\vsub}{\underline{v}}
\newcommand{\vsup}{\overline{v}}
\newcommand{\wsub}{\underline{w}}
\newcommand{\wsup}{\overline{w}}
\newcommand{\ds}{\displaystyle}
\newcommand{\eps}{\varepsilon}

\numberwithin{equation}{section}
\newcommand{\Sect}[1]{\setcounter{equation}{0}\section{#1}\quad}
\newcommand{\dint}{\displaystyle \int}
\newcommand{\tu}{{\tilde u}}
\newcommand{\tw}{{\tilde w}}
%\newcommand{\R}{\mathbb{R}}

\newcommand{\Qed}{\hfill $ \quad \Box $\par\bigskip}




\newtheorem{Remark}{\bf Remark}[section]


\newtheorem{Proposition}[theorem]{Proposition}

\newcommand*{\theorembreak}{\usebeamertemplate{theorem end}\framebreak\usebeamertemplate{theorem begin}}
\usepackage{lipsum}
\usefonttheme{professionalfonts}
\author[nle12@vols.utk.edu]{Van Le }
\institute[UTK] % (optional)
{
	%
		Department of Mathematics\\
	The University of Tennessee, Knoxville\\

}
%\date{May 5, 2023}

\begin{document}
	%\author{Le Nguyen Thuy Van}
	\title[Introduction to machine learning]{An example on data analysis using Scikit-learn package}
	%\subtitle{}
	%\logo{}
	%\institute{}
	%\date{}
	%\subject{}
	%\setbeamercovered{transparent}
	%\setbeamertemplate{navigation symbols}{}
	\frame{\titlepage}
	%\frame
	%{
		%	\begin{multicols}{1}
			%		\tableofcontents
			%	\end{multicols}
		%}
	\frame
	{
		%	\begin{multicols}{1}
			\tableofcontents
			%	\end{multicols}
	}


\section{Main goal}
\frame{\frametitle{Main goal}
\begin{itemize}
	\item Learn to use Scikit-Learn.
	
	\item Predict whether a patient is likely to get a stroke based on inputs such as gender, age, various diseases and smoking status.
	
	\end{itemize}
}

\section{Explore and visualize the data to gain insights}
\begin{frame}
	\centering Explore and visualize the data to gain insights
\end{frame}

\begin{frame}[fragile]
	\frametitle{Downloading data}
\begin{lstlisting}
	stroke_data = pd.read_csv(Path("healthcare-dataset-stroke-data.csv")) 
	stroke_data.head(10)
\end{lstlisting}

\begin{figure}[h]
	\caption{Stroke prediction dataset}
	\centering
	\includegraphics[width=1.0\textwidth]{head.png}
\end{figure}

\end{frame}



\begin{frame}[fragile]
\begin{lstlisting}
	stroke_data.info()
\end{lstlisting}
	
	\begin{figure}[h]
		\caption{Stroke prediction dataset's information}
		\centering
		\includegraphics[width=0.8\textwidth]{info.png}
	\end{figure}
	
\end{frame}



\begin{frame}[fragile]
	\frametitle{A quick look at categorical attributes}
\begin{lstlisting}
stroke_data["NAME_OF_COLS"].value_counts(ascending = True)

>>>gender
Other        1
Male      2115
Female    2994
Name: count, dtype: int64


>>>work_type
Never_worked       22
Govt_job          657
children          687
Self-employed     819
Private          2925
Name: count, dtype: int64
\end{lstlisting}
	
\end{frame}


\begin{frame}[fragile]
	\frametitle{A quick look at categorical attributes}
	\begin{lstlisting}
>>>ever_married
No     1757
Yes    3353
Name: count, dtype: int64
	
>>>Residence_type
Rural    2514
Urban    2596
Name: count, dtype: int64
	
>>>smoking_status
smokes              789
formerly smoked     885
Unknown            1544
never smoked       1892
Name: count, dtype: int64
	
	\end{lstlisting}
	
\end{frame}



\begin{frame}[fragile]
	\frametitle{A quick look at numerical attributes}
\begin{lstlisting}
stroke_data.hist(bins=50, figsize=(12, 8))	\end{lstlisting}
	
	\begin{figure}[h]
		\caption{Histrogram of numerical attributes }
		\centering
		\includegraphics[width=0.7\textwidth]{hist1.png}
	\end{figure}
	
\end{frame}



\section{Create test and training sets}

\begin{frame}
	\centering Create test and training sets
\end{frame}
\begingroup
\footnotesize
\begin{frame}[fragile]
	\frametitle{Create training set and test set}
	We can split the data randomly or based on a certain category. I suspect that each age group has a certain risk, and hence I split the data based on age groups.
	\begin{lstlisting}
		stroke_data["age_groups"] = pd.cut(stroke_data["age"],
		bins=[0., 20, 30, 40, 50, 60, 70, np.inf],
		labels=["0-20", "21-30", "31-40", "41-50", "51-60", "61-70", ">70"])
		stroke_data["age_groups"].value_counts().sort_index().plot.bar(rot=0, grid=True)
	\end{lstlisting}
	
	\begin{figure}[h]
		%\caption{Long-tail attributes}
		\includegraphics[width=0.6\textwidth]{agegroup.png}
	\end{figure}
\end{frame}
\endgroup 


\begin{frame}[fragile]
	\frametitle{Create test and training sets}
	
	\begin{lstlisting}
		from sklearn.model_selection import train_test_split
		x_train, x_test, y_train, y_test = train_test_split(
		stroke_data, stroke_data["stroke"], test_size=0.20, stratify=stroke_data['age_groups'], random_state=42)
	\end{lstlisting}
	
	Let’s see if this worked as expected. Let's look at the income
	category proportions in the test set:
	
	\begin{lstlisting}
		x_train['age_groups'].value_counts() / len(x_train)
		>>>age_groups
		0-20     0.200587
		51-60    0.161204
		41-50    0.144569
		>70      0.138943
		31-40    0.131849
		61-70    0.116194
		21-30    0.106654
		Name: count, dtype: float64
	\end{lstlisting}
\end{frame}

\section{Preparing data}

\begin{frame}
	\centering Preparing data
\end{frame}
\subsection{Clean data}

\begin{frame}[fragile]
	\frametitle{Preparing data}
After looking at the data, we decide to separate them into the $3$ groups 
		\begin{lstlisting}
num_attribs = ["age"]
cat_attribs = ["gender", "ever_married", "work_type", "Residence_type", 
	"smoking_status", "hypertension", "heart_disease"]
log_attribs = ["avg_glucose_level", "bmi"]
		\end{lstlisting}
\end{frame}

\subsubsection{Missing data}
\begin{frame}[fragile]
	\frametitle{Missing data }
We have two options to deal with missing data:
\begin{itemize}
	\item Get rid of the whole attribute.
	\item  Set the missing values to some value (zero, the mean, the median, etc.). This is called imputation.
\end{itemize}

we suspect that "bmi" mat affect the risk of getting strokes. Thus, we decide to imputate the data.
We use the function SimpleImputer as below.
	\begin{lstlisting}
from sklearn.impute import SimpleImputer 
	\end{lstlisting}
There are several ways to impute the data. For example,

\begin{lstlisting}
SimpleImputer(strategy='median')
SimpleImputer(strategy="most_frequent")
\end{lstlisting} 
\end{frame}
\subsubsection{Feature scaling}
\begin{frame}\frametitle{Feature scaling}
	One of the most important transformations is \textit{scaling}. The models do not perform well then the input have very different scales. 
	
	For example, the age attribute is between 0 and 100 while avg\_glucose\_level is above 200.
	In Scikit-Learn, there are two basic transformers called MinMaxScaler and StandardScaler.
	
	\begin{itemize}
		\item MinMaxScaler: This is performed by subtracting the min value and dividing by the difference between the min and the max.
		
		\item StandardScaler: It subtracts the mean value and divides the result by the standard deviation.
	\end{itemize}
\end{frame}

\subsubsection{Encoding}
\begin{frame}\frametitle{Encoding}
	In categorical attribute, each text represents a category. We can choose a function to use from Scikit-Learn package. There are many encoders supported in the package such as OrdinalEncoder, OneHotEncoder.
\end{frame}

\subsection{Transformation pipeline}
\begin{frame}
	\centering Transformation pipelines
\end{frame}
\subsubsection{Numerical attributes}
\begin{frame}[fragile]
	\frametitle{Numerical attributes}
\begin{lstlisting}
num_pipeline = make_pipeline(SimpleImputer(strategy="median"), StandardScaler())
\end{lstlisting}
\end{frame}
\subsubsection{Numerical attributes with long-tail}
\begin{frame}[fragile]
	\frametitle{Long-tail attributes}
	\begin{figure}[h]
		\caption{Long-tail attributes}
		\includegraphics[width=0.6\textwidth]{log.png}
	\end{figure}
	\begin{lstlisting}
		from sklearn.preprocessing import FunctionTransformer
		log_pipeline = make_pipeline(
		SimpleImputer(strategy="median"),
		FunctionTransformer(np.log, feature_names_out="one-to-one"),
		StandardScaler())		
	\end{lstlisting}
\end{frame}

\subsubsection{Categorical attributes}
\begin{frame}[fragile]
	\frametitle{Categorical attributes}

\begin{lstlisting}
cat_pipeline = make_pipeline(SimpleImputer(strategy="most_frequent"), OneHotEncoder(handle_unknown="ignore"))
\end{lstlisting}
\end{frame}






\begin{frame}
	\frametitle{Imbalanced data}
	\begin{wrapfigure}{l}{0.5\textwidth}
		\includegraphics[width=0.4\textwidth]{stroke.png}
	\end{wrapfigure}
	
Notice that we are dealing with imbalanced datasets. The models may have poor performance on the minority class (that had a stroke). We can use Synthetic Minority Over-sampling Technique (SMOTE) in Imbalanced-learn package.
\end{frame}


\begin{frame}[fragile]

\begin{lstlisting}
	from imblearn.over_sampling import SMOTE
	sm = SMOTE(random_state=2)
	x_train_smote, y_train_smote = sm.fit_resample(x_train, y_train.ravel())
\end{lstlisting}

\end{frame}

\section{Some classification models}
\begin{frame}
\frametitle{Classification models}

Here are some models included in Scikit-learn

\begin{itemize}
	\item LogisticRegression
	
	\item DecisionTreeRegressor
	
	\item RandomForestClassifier
	
	\item SVC
	
	\item SVM with Poly Kernel
	
	\item SVM with Gaussian RBF Kernel
\end{itemize}
\end{frame}

\begin{frame}[fragile]
	\frametitle{Accuracy measure}
	\begin{lstlisting}
from sklearn.metrics import accuracy_score
accuracy_score(y_test, y_pred)

from sklearn.model_selection import cross_val_score
cross_val_score(estimator = NAME_OF_MODELS, X = x_train_smote, y = y_train_smote, cv = 3) 
	\end{lstlisting}
\end{frame}



\begin{frame}
	\frametitle{Compare accuracy of models}
	Data is cleaned in the pipelines:
		\begin{center}
		\begin{tabular}{ | c | c| c | } 
			\hline
			& Accuracy score & K-fold cross-validation \\ 
			\hline
			LogisticRegression & $83.37 \%$ & $82.41 \%$ \\ 
			\hline
			%	DecisionTreeRegressor & $9080 \% $ & cell6 \\ 
			%		\hline
			RandomForestClassifier & $91.59 \%$ & $93.82 \%$ \\ 
			\hline
			SVC & $84.76 \%$ & $87.91 \%$ \\ 
			\hline
			SVM with Poly Kernel & $85.03 \%$ & $88.41 \%$  \\ 
			\hline
			SVM with Gaussian RBF Kernel & $86.59 \%$ & $90.95 \%$  \\ 
			\hline
		\end{tabular}
	\end{center}
	Data is cleaned before and in the pipelines:
		\begin{center}
		\begin{tabular}{ | c | c| c | } 
			\hline
			& Accuracy score & K-fold cross-validation \\ 
			\hline
			LogisticRegression & $95.79 \%$ & $96.20 \%$ \\ 
			\hline
			%	DecisionTreeRegressor & $9080 \% $ & cell6 \\ 
			%		\hline
			RandomForestClassifier & $95.11 \%$ & $96.92 \%$ \\ 
			\hline
			SVC & $95.69 \%$ & $94.99 \%$ \\ 
			\hline
			SVM with Poly Kernel & $95.69 \%$ & $96.50 \%$  \\ 
			\hline
			SVM with Gaussian RBF Kernel & $91.88 \%$ & $94.59 \%$  \\ 
			\hline
		\end{tabular}
	\end{center}

\end{frame}


\begin{frame}
	
\end{frame}



\end{document}