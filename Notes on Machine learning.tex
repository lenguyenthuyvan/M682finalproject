\documentclass[11pt,a4paper]{article}
\usepackage{color,fancyhdr,ifthen,amssymb,amsfonts,amsmath,graphicx,amsthm}
\usepackage[english]{babel}
\usepackage{setspace}
\setstretch{1.25}
\usepackage{tikz}
\usepackage{ragged2e}
\usepackage{blindtext}

\newtheorem{theorem}{Theorem}[section]
\newtheorem{corollary}{Corollary}[theorem]
\newtheorem{lemma}[theorem]{Lemma}
\newtheorem{remark}{Remark}
\newtheorem{definition}{Definition}[section]
\newtheorem{example}{Example}[subsection]
\newtheorem{ex}{Exercise}
\newtheorem{question}{Question}
\numberwithin{question}{subsection}

\newtheorem{proposition}{Proposition}[section]
\newcommand{\R}{\mathbb{R}}
\newcommand{\N}{\mathbb{N}}
\newcommand{\Z}{\mathbb{Z}}
\newcommand{\Q}{\mathbb{Q}}
\newcommand{\C}{\mathbb{C}}
\newcommand{\D}{\mathbb{D}}
\newcommand{\p}{\partial}
\newcommand{\A}{\mathcal{A}}
\newcommand{\F}{\mathcal{F}}
\newcommand{\B}{\mathcal{B}}
\newcommand{\M}{\mathcal{M}}

\def\Xint#1{\mathchoice
	{\XXint\displaystyle\textstyle{#1}}%
	{\XXint\textstyle\scriptstyle{#1}}%
	{\XXint\scriptstyle\scriptscriptstyle{#1}}%
	{\XXint\scriptscriptstyle\scriptscriptstyle{#1}}%
	\!\int}
\def\XXint#1#2#3{{\setbox0=\hbox{$#1{#2#3}{\int}$ }
		\vcenter{\hbox{$#2#3$ }}\kern-.6\wd0}}
\def\ddashint{\Xint=}
\def\dashint{\Xint-}

%%SQUARECASESA%%%
\DeclareMathOperator{\rank}{rank}
\makeatletter
\newenvironment{sqcases}{%
	\matrix@check\sqcases\env@sqcases
}{%
	\endarray\right.%
}
\def\env@sqcases{%
	\let\@ifnextchar\new@ifnextchar
	\left\lbrack
	\def\arraystretch{1.2}%
	\array{@{}l@{\quad}l@{}}%
}
\makeatother

\pagestyle{fancy}
\setlength{\topmargin}{-.5in}
\setlength{\textheight}{9in}
\setlength{\oddsidemargin}{0in}
\setlength{\evensidemargin}{0in}
\setlength{\textwidth}{6.5in}
\setlength{\headwidth}{\textwidth}
\parindent=0in
\newcounter{questionNumber}
\setcounter{questionNumber}{1}
\newcommand{\headandfoot}[3]{\lhead{#1}\chead{#2}\rhead{\ifthenelse{\isodd{\thepage}}{ {\hspace{.25in}}}{}}
	\lfoot{ }\cfoot{\thepage}\rfoot{#3}}
\linespread{1.5}
%\newcommand{\question}[6]{\noindent\thequestionNumber.\quad %#1\\[.1in]
	%(a) #2 \\
	%(b) #3 \\
	%(c) #4 \\
	%(d) #5 \\
	%(e) #6
	\vspace{.25in}\addtocounter{questionNumber}{1}
	\headandfoot{Machine learning notes}{}{}


\begin{document}
\section{NOTES ON MACHINE LEARNING}

\subsection{Approaching a data set}

Here are some main steps to approach a problem.
\begin{itemize}
	\item Look at the big picture. 
	\begin{itemize}
		\item Identify the objectives/goals.
		
		\item Select a performance measure.
		
		\item Check the assumptions to identify the task. Classification or regression.
	\end{itemize} 
	\item Get the data.
	
	\begin{itemize}
		\item Download the data.
		
			\item Quick look at the data structure.
		
		\item Create a test set.
	\end{itemize}
	
	\item Explore and visualize the data to gain insights.
	
	\begin{itemize}
	\item  Look for correlations.
	
	\item   Experiment with Attribute Combinations.
	\end{itemize}
	
	\item Prepare the data for machine learning algorithms.
	
	\begin{itemize}
		\item Clean the data.
		
		\item Handling Text and Categorical Attributes.
		
		\item  Feature Scaling and Transformation.
		
		\item  Custom Transformers.
		
		\item  Transformation Pipelines.
	\end{itemize}
	 \item Select a model and train it.
	 \begin{itemize}
	 	\item  Train and Evaluate on the Training Set
	 	
	 	\item   Better Evaluation Using Cross-Validation
	 \end{itemize}
	\item Fine-tune your model.
	\begin{itemize}
		\item Grid search.
		
		\item  Randomized Search.
		
		\item Ensemble Methods.
		
		\item Analyzing the Best Models and Their Errors.
		
		\item Evaluate Your System on the Test Set.
	\end{itemize}
	\item Present your solution.
	
	\item Launch, monitor, and maintain your system.
	
	
\end{itemize}
	
	
	
	\section{Projects }

\subsection{Strokes prediction}
\subsubsection{Big picture}

\begin{itemize}
	\item Objectives/goals
	
	I want to predict whether a patient is likely to get stroke based on the input parameters like gender, age, various diseases, and smoking status. 
	
	\item  Select a performance measure.
	
		\item Check the assumptions to identify the task. Classification or regression.
	
	For practicing purpose, I will try to use both regression and classification.
\end{itemize}
	
\end{document}